\chapter{Introduction}
\section {Type systems}
Type systems are well known concept for programming languages.
They allow programmers to write down simple specifations to functions that can easily be checked by the type checker of the program, thus leading to less errors in programs.
Thus, more expressive type systems that allow a better specification of the function is desirable.
Haskell \cite{haskell} has an already quite expressive type system.
Another attempt to improve software reliability is the use of computer aided formal reasoning.
Here, the user tries to proof properties of his software with the help of the computer.
Depentend types provide the underlying link between these two efforts.
For one, many theorem provers like Coq \cite{coq} , Lego \cite{pollack94theory} , Twelf \cite{pfenning99system}.
are internally based on depentend types.
On the other side, a programming language with a depentent type system would be quite powerfull for specification.
Through the Curry-Howard Isomorphism, proofs are treated as functions, thus propositions correspond to types.
Recently , languages like Agda \cite{norell:thesis}, Cayenne \cite{augustsson98cayenne} and Epigram \cite{epigram} try to combine both aspects of depentend types.
Programming was added to Coq.

\section {Termination checking}
Termination checking is especially important for theorem proving.
For one, we don't want to wait very long for type checking, only to suspect that there is an
an inifite computation going on.
More importantly, one might easily have a proof as a function that that looks correct, but is not because it is not total, i.e is not definied on some input values.
A function defined by the pattern matching equation $ f x = f x $ can be seen as a proof for every proposition.
("Why ? just because ! ")
One common criterion for recursive functions is : Every recursive call has to happen on a smaller argument.
If the arguments are well-founded, this will 
For multiple arguments and mutally recursive functions, the size-change principle can be used.
One other mehod that has been proposed are sized types.
Types are decorated with a size annotation to help see the system that arguments are getting smaller.
The reader may wonder if depentend types are not already enough to specify such properities.
And indeed, they nearly are.
\section{Infite Objects}
Many objects for todays computers are potentially infite - a stream of network packages for example.
So 
While we can't expect that we can analyse all packets, what we want to guarantee that during finite time,
new packages will arrive. This is called productivity for streams.
Productivity of an inifite objects means we can look at any part in finite time.
As it turns out, the sized type is suited to handle productivy for our purposes.
\section {Curry Howard Isomorphism}
The Curry Howard Isomorphism relates theorem proving to functional programming.
\section{Agda}
Agda is a familiy of theorem provers / programming languages based primairly on the 
orinal definiton of type theory by Martin-L\"öf \cite{mart84}.
The current Agda2 version is developed by Ulf Norell.
Agda2 allows general recursive functions to be defined by pattern matching.
Our goal was to study the addition of a size type to a system in the style of Agda, and to
extend it to inifite objetcs.

\section{Overview}
In chapter 2 we introduce our language MiniAgda and present type checking.
In chapter 3 we add a termination checker based on structural recursion and the size-change principle.
In chapter 4, we add a size type to enable even more valid definitons to be accepted as total by the system.
We define valied uses of the size type.
In chapter 5, we extend the language to coinductive objects with corecursive functions.
We again present 
In chapter 6, we detail the implementation of the resulting system.
Finally in chapter 7, we give an outlook for improvements.





