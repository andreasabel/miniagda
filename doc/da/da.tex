\documentclass[a4paper,11pt]{report}
\usepackage[T1]{fontenc}
\usepackage{amsmath,amsthm, amssymb}
\usepackage{graphicx}
\usepackage{latexsym}
\usepackage{pxfonts}
\usepackage[german,english]{babel} 

\pagestyle{headings}

\newtheorem{thm}{Theorem}[section]
\newtheorem{cor}[thm]{Corollary}
\newtheorem{lem}[thm]{Lemma}

\newenvironment{bsp}
{\begin{list}{}{
\setlength{\leftmargin}{0cm}
\setlength{\rightmargin}{0cm}
}\item}{
\end {list}
}


\theoremstyle{remark}
\newtheorem{rem}[thm]{Remark}

\newcommand{\vex}[1]{\overrightarrow{#1}} 
\newcommand{\ves}[1]{{#1}^{\ast}} 

\newcommand{\vsp}{\:\vert\:}
\newcommand{\Set}{\mathsf{Set}}
\newcommand{\s}{\mathsf{s}\,}
\newcommand{\Size}{\mathsf{Size}}
\newcommand{\EPi}[2]{( #1 : #2 ) \rightarrow} 
\newcommand{\data}{\mathrm{data}\:}
\newcommand{\sized}{\mathrm{sized}\:}
\newcommand{\fun}{\mathrm{fun}\:}
\newcommand{\codata}{\mathrm{codata}\:}
\newcommand{\cofun}{\mathrm{cofun}\:}
\newcommand{\clet}{\mathrm{let}\:}
\newcommand{\ELet}[3]{\mathrm{let}\: #1 : #2 = #3 \: \mathrm{in}\:} 
\newcommand{\mutual}{\mathrm{mutual}\:}
\newcommand{\lam}[1]{\lambda\: #1 .\:}

\newcommand{\inacc}[1]{\underline{#1}}

\newcommand{\id}{\mathsf{id}\:}
\newcommand{\Nat}{\mathsf{Nat}\:}
\newcommand{\zero}{\mathsf{zero}\:}
\newcommand{\suc}{\mathsf{succ}\:}
\newcommand{\add}{\mathsf{add}\:}
\newcommand{\addt}{\mathsf{add_2}\:}

\newcommand{\NatP}{\mathsf{NatPair}\:}
\newcommand{\np}{\mathsf{np}\:}
\newcommand{\addp}{\mathsf{add_p}\:}

\newcommand{\even}{\mathsf{even}\:}
\newcommand{\odd}{\mathsf{odd}\:}

\newcommand{\Bool}{\mathsf{Bool}}
\newcommand{\ttt}{\mathsf{tt}\:}
\newcommand{\fff}{\mathsf{ff}\:}

\newcommand{\List}{\mathsf{List}\:}
\newcommand{\nil}{\mathsf{nil}\:}
\newcommand{\cons}{\mathsf{cons}\:}

\newcommand{\Term}{\mathsf{Term}\:}
\newcommand{\abs}{\mathsf{abs}\:}
\newcommand{\funt}{\mathsf{Fun}\:}
\newcommand{\func}{\mathsf{fn}\:}
\newcommand{\omegad}{\mathsf{omega}\:}
\newcommand{\app}{\mathsf{app}\:}

\newcommand{\Tree}{\mathsf{Tree}\:}
\newcommand{\leaf}{\mathsf{leaf}\:}
\newcommand{\node}{\mathsf{node}\:}

\newcommand{\vVec}{\mathsf{Vec}\:}
\newcommand{\head}{\mathsf{head}\:}
\newcommand{\rev}{\mathsf{rev}\:}
\newcommand{\reva}{\mathsf{rev1}\:}
\newcommand{\revb}{\mathsf{rev2}\:}


\newcommand{\flatt}{\mathsf{flat}\:}

\newcommand{\Eq}{\mathsf{Eq}\:}
\newcommand{\refl}{\mathsf{refl}\:}

\newcommand{\prof}{\mathsf{proof}\:}
\newcommand{\tprof}{\mathsf{proof2}\:}
\newcommand{\eqsucc}{\mathsf{eqsucc}\:}

\newcommand{\Bad}{\mathsf{Bad}\:}
\newcommand{\bad}{\mathsf{bad}\:}
\newcommand{\ok}{\mathsf{ok}\:}

\newcommand{\ProdT}{\mathsf{Prod}\:}
\newcommand{\prd}{\mathsf{prod}\:}
\newcommand{\pra}{\mathsf{pr_1}\:}
\newcommand{\prb}{\mathsf{pr_2}\:}
\newcommand{\quicksort}{\mathsf{quicksort}\:}
\newcommand{\qsapp}{\mathsf{qsapp}\:}
\newcommand{\pivot}{\mathsf{split}\:}
\newcommand{\lleq}{\mathit{leq}\:}
\newcommand{\ite}{\mathsf{ite}\:}

\newcommand{\Ord}{\mathsf{Ord}\:}
\newcommand{\ozero}{\mathsf{ozero}\:}
\newcommand{\olim}{\mathsf{olim}\:}
\newcommand{\addOrd}{\mathsf{addOrd}\:}

\newcommand{\Stream}{\mathsf{Stream}\:}
\newcommand{\tail}{\mathsf{tail}\:}
\newcommand{\nth}{\mathsf{nth}\:}

\newcommand{\wkSNat}{\mathsf{weakSNat}\:}
\newcommand{\wkSNatb}{\mathsf{badSNat}\:}
\newcommand{\wkStream}{\mathsf{weakStream}\:}


\newcommand{\minus}{\mathsf{minus}\:}
\renewcommand{\div}{\mathsf{div}\:}

\newcommand{\SNat}{\mathsf{SNat}\:}
\newcommand{\Maybe}{\mathsf{Maybe}\:}
\newcommand{\nothing}{\mathsf{nothing}\:}
\newcommand{\just}{\mathsf{just}\:}
\newcommand{\shift}{\mathsf{shift}\:}
\newcommand{\shiftcase}{\mathsf{shift\_case}\:}
\newcommand{\inc}{\mathsf{inc}\:}
\newcommand{\lop}{\mathsf{loop}\:}
\newcommand{\lopcase}{\mathsf{loop\_case}\:}
\newcommand{\diverge}{\mathsf{diverge}\:}

\newcommand{\zeroes}{\mathsf{zeroes}\:}
\newcommand{\unp}{\mathsf{unp}\:}
\newcommand{\zipWith}{\mathsf{zipWith}\:}
\newcommand{\fibs}{\mathsf{fib'}\:}
\newcommand{\fib}{\mathsf{fib}\:}
\newcommand{\fibf}{\mathsf{fib4}\:}

\newcommand{\Empty}{\mathsf{Empty}\:}
\newcommand{\BadNat}{\mathsf{BadNat}\:}
\newcommand{\foo}{\mathsf{foo}\:}

\newcommand{\addWith}{\mathsf{addWith}\:}

\newcommand{\ispd}{\mathsf{ISP}\:}
\newcommand{\spd}{\mathsf{SP}\:}
\newcommand{\putd}{\mathsf{put}\:}
\newcommand{\getd}{\mathsf{get}\:}
\newcommand{\isp}{\mathsf{isp}\:}
\newcommand{\ieat}{\mathsf{ieat}\:}
\newcommand{\eat}{\mathsf{eat}\:}

\newcommand{\adder}{\mathsf{adder}\:}
\newcommand{\cadder}{\mathsf{cadder}\:}
\newcommand{\iadder}{\mathsf{iadder}\:}

\newcommand{\eqo}{\mathsf{eq_0}\:}
\newcommand{\zeroest}{\mathsf{zeroes_2}\:}
\newcommand{\beqd}{\mathsf{Beq}\:}
\newcommand{\beq}{\mathsf{beq}\:}
\newcommand{\eqt}{\mathsf{eq_1}\:}

\newcommand{\ra}{\rightarrow}

\newcommand{\spc}{\hspace*{3mm}}
\newcommand{\spcx}{\hspace*{10mm}}
\newcommand{\vs}{\vspace*{1mm}}

\newcommand{\mugda}{\textsf{Mugda }}

\newtheorem{definition}{Definition}

\begin{document}
\begin{titlepage}
\begin{center}


\vspace*{-2cm}

\selectlanguage{german} 
{\Huge INSTITUT F\"UR INFORMATIK\\[1mm]} % 
DER LUDWIG--MAXIMILIANS--UNIVERSIT\"AT M\"UNCHEN\\

\vspace*{1cm}

\includegraphics[width=0.45\textwidth]{siegel.pdf}

\vspace*{2cm}

{\LARGE \textbf{Diplomarbeit}}\\

\vspace{2.0cm}
{\LARGE \textbf{TERMINATION CHECKING FOR A}}\\
\vspace*{3mm}
{\LARGE \textbf{DEPENDENTLY TYPED LANGUAGE}}\\

\vspace{2cm}

\Large{Dezember 2007}

\vspace{1.5cm}

  \begin{Large}
  \begin{tabular}{rl}
      Autor: &Karl Mehltretter\\
      Aufgabensteller: & Prof. Dr. Martin Hofmann\\
      Betreuer: & Dr. Andreas Abel\\
  \end{tabular}
  \end{Large}
\end{center}
\
\end{titlepage} 

\chapter*{Erkl\"arung}
Hiermit erkl\"are ich, dass ich die vorliegenden Diplomarbeit selbst\"andig verfasst und keine anderen als die angegebenen Hilfsmittel verwendet habe.
\\
\\
\\
\\
\\
\rule{6cm}{0.4pt}\\
{\Large Karl Mehltretter}\\

\chapter*{Danksagung}
Diese Diplomarbeit enstand am tcs Lehrstuhl der Universit\"at M\"unchen.
Ich danke vor allem meinem Betreuer Andreas Abel f\"ur seine tolle Unterst\"utzung.

Desweiteren m\"ochte ich mich bei meinen Eltern, bei Sandrine, bei Karin und Thomas, bei Andy, Robert und bei Christian f\"ur ihren R\"uckhalt w\"ahrend dieser anstrengen Zeit bedanken.



\title{Termination Checking for a dependently typed language}
\author{Karl Mehltretter}
\date{today}
\renewcommand{\abstractname}{Inhalt}
\begin{abstract}
Abh\"angige Typen werden als Basis f\"ur viele interaktive Theorembeweiser eingesetzt und es gibt seit einiger Zeit Bem\"uhungen, sie f\"ur allgemeine funktionale Programmiersprachen zu verwenden.
Eine Sprache mit abh\"angigen Typen erlaubt es, in einem gemeinsamen Rahmen sowohl Programme zu schreiben als auch Beweise zu formalisieren.

Das aus der Programmierung bekannte \emph{pattern matching} wurde als Mittel zur Definition von
rekursiven Funktionen in diesen Systemen vorgeschlagen.
Zwar ist pattern matching intuitiv und ausdrucksstark, doch muss sicher\-gestellt werden, dass die so definierten Funktionen total sind.

Ein Aspekt einer totalen Funktion ist Terminierung: Jeder Auswertung der Funktion muss in endlicher Zeit m\"oglich sein.  Unendlich Objekte k\"onnen unterst\"utzt werden, indem jeweils nur Teile davon berechnet werden.
G\"ultige Definitionen f\"ur unendliche Objekte m\"ussen \emph{produktiv} sein.

Da das Halteproblem unentscheidbar ist, kann das Ziel nur sein, m\"o\-glichst viele Definition als terminierend zu erkennnen. Es gibt verschiedene Ans\"atze zum Uberpr\"ufen der Terminierung.
Oft k\"onnen Funktionen, deren rekursive Aufrufe einem Schema wie folgen,
 vom System zugelassen werden. Das \emph{size change principle} umfasst einige dieser Schemata. 

Ein weiterer Vorschlag ist der Einsatz von \emph{sized types}. Information \"uber die Gr\"o\ss e von Argumenten wird im Typsystem verfolgt, um diese f\"ur das Erkennen der Terminierung zu nutzen.   

Im Rahmen dieser Diplomarbeit wurde das System \mugda entwickelt und wird nachfolgend vorgestellt.
Es basiert auf Typtheorie nach Martin-L\"of und unterst\"utzt induktive und co-induktive Typen. Funktionen werden  durch pattern matching definiert.
Das Kriterium f\"ur Terminierung basiert auf dem \emph{size change principle}.
Au\ss erdem stelle die Sprache Elemente bereit, um dem Benutzer die Verwendung von \emph{sized types} zu erm\"oglichen.
\end{abstract}
\newpage

\selectlanguage{english} 
\renewcommand{\abstractname}{Abstract}
\begin{abstract}
Dependent types have been used at the core of many proof assistants, and more recently there haven been efforts to extend their use to functional programming languages. 
Dependent type theories allow programming and reasoning in one common framework.

The \emph{pattern matching} notation, known from traditional programming, has been proposed for defining recursive functions in such systems.
While pattern matching is intuitive and powerful, it has to be ensured that the defined functions are total. 

One aspect of totality is termination: each evaluation of a function must be able to be computed in finite time.
Infinite objects can be added to the language by computing only finite parts of them as necessary.
Valid definitions of infinite objects need to be \emph{productive}. 

As the Halteproblem is undecidable, the goal can only be to accept as many valid definitions as possible.
There are many approaches to ensure termination.
One is to allow the definition of functions where the recursive calls follow a certain scheme.
The \emph{size change principle} subsumes some of these schemes.

With the use of \emph{sized types}, information about the height of arguments is tracked in the type system and can be used to recognize more definitions as terminating.  

For this thesis, the system \mugda was developed and will subsequently be described. It is based on \emph{Martin-L\"of type theory} and supports inductive and co-inductive types. Functions are defined by pattern matching. Its termination criterion is based on the size-change principle. In  addition, the language provides elements to enable the use of sized types.

\end{abstract}
\tableofcontents
\chapter{Introduction}

\section {Computer Aided Theorem proving}
Formal proofs with the help of a computer.
\section {Curry Howard Isomorphism}
The Curry Howard Isomorphism relates theorem proving to functional programming.
\section {Termination checking}
Termination checking is especially important for theorem proving.



\documentclass{article}

\usepackage{hcar}

\begin{document}

% MiniAgda-AM.tex
\begin{hcarentry}[section,updated]{MiniAgda}
\report{Andreas Abel}%05/2014
\status{experimental}
\makeheader

MiniAgda is a tiny dependently-typed programming language in the style
of Agda~\cref{agda}. It serves as a laboratory to test 
potential additions to the
language and type system of Agda. MiniAgda's termination checker is a
fusion of sized types and size-change termination and supports
coinduction. Bounded size quantification and destructor
patterns for a more general handling of coinduction.
Equality incorporates eta-expansion at record and
singleton types. Function arguments can be declared as static; such
arguments are discarded during equality checking and compilation.

MiniAgda is now hosted on \url{http://hub.darcs.net/abel/miniagda}.
 
MiniAgda is available as Haskell source code on hackage and compiles with GHC
6.12.x -- 7.8.2. 

\FurtherReading
  \url{http://www.cse.chalmers.se/~abela/miniagda/}
\end{hcarentry}


\end{document}

\chapter{Conclusion}
We hope this work provides an easy access to dependent types and the sized type principle.
The pattern matching of \mugda should be more accessible than the traditional fixed-point operator
notation used in a lot of previous presentations of sized types.
We think that the \mugda language demonstrates the usefulness of sized types.
To that end, we have stretched Coquand's simple algorithm to a quite usable system.
\subsubsection{Inference of sizes}
The possibility to infer all size expressions automatically should be explored.
The system $\mathsf{CIC\textasciicircum}$ described in \cite{bgp:lpar06} can handle the inference of size annotations. 

Applying inference to our system would change the system quite a bit.
Size annotations would not be available to the user and the size type would likely loose its first order status. Every data type would internally be represented as a sized type. The termination criterion could be simplified, where the size-change principle has to be only applied to all inferred size arguments.
\subsubsection{Implicit arguments}
At run-time, the size arguments all end up being $\infty$.
Thus, it would be wasteful that this payload is kept around during run-time. 
the removal of ballast that is not needed at run-time is ongoing research (\cite{miquel01implicit,DBLP:conf/types/BradyMM03}).
As lists, vectors and sized list carry the same payload, conversion functions can be written by the user to change from one representation to the other. At run-time, then these conversion routines could be safely removed because it would amount to an identity function.
\subsubsection{More mutual definitions}
We could allow more kinds of mutual definitions.
Defining a $\fun$ together with a $\data$ type amounts to the so called inductive-recursive definitions (\cite{dybjer01indexed}).
Also, mixed inductive/co-inductive definitions would be interesting. We investigated this approach and some adjustment of the admissibility of sizes seems to be necessary to handle such definitions.
\subsubsection{Higher order sub-typing and universes }
The sub-typing relation could be extend to higher order sub-typing.
Polarized sub-typing (\cite{steffen:phd}) could be explored.
The universe hierarchy also often has sub-typing ($ \mathsf{Set_i} \leq \mathsf{Set_{i+1}}$), which has to be integrated with the sub-typing on sized types.
\subsubsection{Future of sized types}
We think it could be worthwhile to integrate a sized type approach into a full system like Agda2, especially with the extension to co-inductive types in mind. Productivity is handled quite naturally with a sized type approach. 
Furthermore, there are plans to translate the full Agda2 to a simpler core language (\cite{mini-tt}), which can be more easily justified. Once again, a sized type approach could be looked at for this core language.


\appendix
\chapter{Mugda implementation}
$\mugda$ was implemented in the function language Haskell \cite{haskell}, using the Glasgow Haskell compiler (GHC) \cite{jones93glasgow}.

\section{Source file listing}
\begin{itemize}
\item
\texttt{Lexer.x} : the alex lexer file
\item
\texttt{Parser.y} : the happy parser file
\item
\texttt{Concrete.hs} : concrete syntax produced by the parser
\item
\texttt{TraceError.hs} : provides the user a trace when an error occurs
\item
\texttt{ScopeChecker.hs} : turns concrete into abstract syntax
\item
\texttt{Abstract.hs} : produced by the scope-checker, used during type-checking
\item
\texttt{Values.hs} : defines values , evaluation, signature , type-check monad
\item
\texttt{TypeChecker.hs} : type-checking with admissibility
\item
\texttt{Termination.hs} : syntactic termination check
\item
\texttt{Completeness.hs} : size pattern completeness check
\item
\texttt{SPos.hs} : strict positivity checker
\item
\texttt{Main.hs} : the main module
\item
\texttt{example} directory: example input files
\item
\texttt{Makefile} : for compilation
\end{itemize}
\section{Usage}

\mugda as presented was pretty much directly transfered to ASCII syntax:
\begin{itemize}
\item
lists of constructors and clauses are grouped with brackets \texttt{\{} \texttt{\}} and separated with semicolon \texttt{;}
\item
$\EPi{x}{A}{B}$ is written \texttt{(x : A) -> B }
\item
$A \ra B$ is written \texttt{A -> B}
\item
$\ELet{x}{A}{e}{f}$ is written \texttt{let x : A = e in f}
\item
$ \lam{x}e$ is written \verb+\x -> e +
\item
$ \inacc{e}$ is written \texttt{.e} 
\item
$ \infty$ is written \texttt{\#} 
\item
$\s$ is written \texttt{\$}
\item
one line comments are prefixed by \verb+--+
\item
multi-line comments are put between \verb+{-+ and \verb+-}+
\end{itemize}
a \texttt{let} declaration can be prefixed with \texttt{eval}. Then the value will be evaluated after type checking is done.
As an example showing most of the syntactical features, here is the Fibonacci stream example (\texttt{examples/fib.ma}) in text format:
\begin{verbatim}
data Nat : Set {
  zero : Nat;
  succ : Nat -> Nat 
}

fun add : Nat -> Nat -> Nat {
  add zero = \y -> y;
  add (succ x) = \y -> succ (add x y)
}

sized codata Stream : Size -> Set {
  cons : (i : Size) -> Nat -> Stream i -> Stream ($ i)
}
 
fun tail : Stream # -> Stream # {
  tail (cons .# x xs) = xs
}

fun head : Stream # -> Nat {
  head (cons .# x xs) = x
}

fun nth : Nat -> Stream # -> Nat {
  nth zero xs = head xs;
  nth (succ x) xs = nth x (tail xs) 
}

let 1 : Nat = (succ zero)

cofun fib' : (x : Nat ) -> (y : Nat ) 
               -> (i : Size ) -> Stream i {
  fib' x y ($ i) = cons i x (fib' y (add x y) i)
} 

-- fib = 1, 1, 2, 3, 5 , 8 ...
let fib : Stream # = (fib' 1 1 #)

let 4 : Nat = (succ (succ (succ 1)))

-- fib(4) = 5 
eval let fib4 : Nat = nth 4 fib 
\end{verbatim}
Running \texttt{Main examples/fib.ma} yields the console output:
\begin{verbatim}
***** Mugda v1.0 *****
--- scope checking ---
--- type checking ---
--- evaluating ---
fib4 evaluates to (succ (succ (succ (succ (succ zero)))))
\end{verbatim}

\section{Some implementation details}
The alex \cite{alex} and happy \cite{happy} tools were used to generate lexer and parser for $\mugda$.
Most of the Haskell code is monadic, where \emph{monad transformers} \cite{Grabmueller2006MonadTransformers} are used to keep the signature in a state monad, to provide I/O and tracing of errors. The execution of $\mugda$ can be broken into 4 stages:
\paragraph*{Parsing}
The input file is parsed into \emph{concrete syntax}.
\paragraph*{Scope-Checking}
As mentioned in chapter 3, scope-checking is the first step after parsing.
During parsing, it is not known whether an identifier is variable or constructor etc.
Scope-checking produces \emph{abstract syntax} where all identifiers are categorized, if the $\mugda$ program is well-scoped.
Also some syntactic tests like checking linearity of patterns are already done during this stage.
\paragraph*{Type-Checking}
Every declaration is type-checked.
For mutual declarations, the type-checker also checks admissibility and finally invokes the termination-checker. 
\paragraph*{Evaluation}
For all declarations of the form \verb+eval let l : A = e+, \verb+e+ is evaluated and this value is displayed.
\chapter{Bibilography}
\bibliographystyle{alpha}
\bibliography{da}



\end{document}