\chapter{Conclusion}
We hope this work provides an easy access to depented types and the sized type principle.
The pattern mathching of MiniAgda is in our opinion more accesible than the traditional fixpoint operator
notation of previous articles about sized types.
We think that the MiniAgda language demonstrates the usefullnes of sized types.
They are quite elegent for coinductive definitions.
\section{Inference}
We should investigate the possibility of infering the size arguments.
This would entail completly hiding the size type from the user.
Barthe has already developed inferences algorithm for his sized version of CIC.
\section{Proofs of correctness}
Obviosuly, proofing the correctness of the admissibilty criterion is important.
This should involve a model construction, which is not easy for a language with inductive families.
\section{Polarized Subtyping}
The subtyping could be properly extend to higher order subtyping.
The use of polarities could be used.
