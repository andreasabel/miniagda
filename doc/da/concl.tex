\chapter{Conclusion}
We hope this work provides an easy access to depented types and the sized type principle.
The pattern matching of MiniAgda is in our opinion more accessible than the traditional fix-point operator
notation of previous articles about sized types.
We think that the MiniAgda language demonstrates the usefulness of sized types.
They are quite elegant for co-inductive definitions.
\section{Inference}
We should investigate the possibility of inferring the size arguments.
This would entail completely hiding the size type from the user.
Barthe has already developed inferences algorithm for his sized version of CIC.
\section{Proofs of correctness}
Obviously, proofing the correctness of the admissibility criterion is important.
This should involve a model construction, which is not easy for a language with inductive families.
\section{Polarized Sub-typing}
The sub-typing could be properly extend to higher order sub-typing.
The use of polarities could be used.
